\chapter{SYSTEM DESCRIPTION}
The purpose of the Chapter 3 is to describe the process of designing your project. The detail should be sufficient so that the reader can easily understand what was done. A brief summary of the unique approach your group used to solve the problem should be given, also including a concise introduction to theory or concepts used to analyze and calculate. To improve clarity of presentation, this section may be further divided into subsections as below:

%%%%%%%%%%%%%%%%%%%%%%%%%%%%%%%%%%%%%%%%%%%%%%%%%%%%%%
\section{Block Diagram of the System}
\begin{itemize}
\item Present an overview of the system’s architecture through a block diagram. Each block should represent a key component or module of your project
\item Describe the purpose and functionality of each block, highlighting their
relationships and interactions.
\end{itemize}

%%%%%%%%%%%%%%%%%%%%%%%%%%%%%%%%%%%%%%%%%%%%%%%%%%%%%%
\section{Design of Each Block and Select the Best Alternative}
\begin{itemize}
\item Here, delve into the design details of each individual block from the block diagram. Discuss different design alternatives considered and the rationale behind selecting the final design.
\item If any unique approaches were used, elaborate on them. Explain how each block's design contributes to the overall functionality of the system.
\end{itemize}

%%%%%%%%%%%%%%%%%%%%%%%%%%%%%%%%%%%%%%%%%%%%%%%%%%%%%%
\section{Testing of Each Block}
\begin{itemize}
\item This subsection covers the comprehensive testing procedures conducted for each block.
\item Please note that this subsection can be removed/modified depending on the specific project.
\end{itemize}


%%%%%%%%%%%%%%%%%%%%%%%%%%%%%%%%%%%%%%%%%%%%%%%%%%%%%%
\section{System Implementation}
\begin{itemize}
\item Discuss the practical steps taken to translate the conceptual design into tangible
components. Considerations for scalability, efficiency, and real-world
implementation are detailed.
\end{itemize}
